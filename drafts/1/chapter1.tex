% !TEX root  = /main.tex %
\chapter{Introduction - Overview, Poetics and Histories}
\section{Overview}

\section{Poetics}
Motion plays a large part in the subjective experience of music. Foremost we are aware of the motions of the performers' bodies as they elicit sounds from their instruments. String players walk their fingers up and down the neck and draw their bow across the strings. Wind players breathe and configure their fingers. We see pianists hands dance up and down the keyboard. Performers also use their bodies as an expressive element in motions that are secondary to sound production. They create all manner of gesture: swaying, nodding; tapping their foot to the beat. These motions enhance and reinforce the music. On the receiving end, listeners make sense of music through the motion their bodies. The psychology field of Embodied Music Cognition studies this phenomenon.  

\section{Histories}
There are previous works whose consideration may be useful in understanding the composition practices of \compositionTitle.


% THESIS:
